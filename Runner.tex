\title{Filesystem virtualization without containers}

\abstract{
Containers became a popular way to distribute programs in a portable fashion and to run them
under strict security constraints. However, thanks to their popularity and relatively ease
of use, users often create full-fledged containers to run mundane executables. That renders to
several complete Linux distributions being stored on the same device and brings considerable
overhead to the load and execution of those programs. This paper presents the design and
implementation of a virtualization tool that has been integrated into a full Linux-based
distribution. That tool multiplexes the root filesystem according to the dependencies of programs
executed without need for management of containers or images. Further, it seamlessly handles
dependencies managed by third-party programming language modules like PIP and LuaRocks by treating
them as first-class programs under the multiplexed filesystem tree.
}

\section{Introduction}
Dependency hell is an old problem that haunts users wishing to install programs depending on
different and often conflicting versions of a given package. That is the case of non-versioned
libraries that attempt to overwrite files from a previous installation, effectively preventing
concurrent versions of that library from coexisting. The same is true for packages whose executable
names does not change across releases; unless the user renames the existing executable file names
prior to the installation of a new version it is not possible to keep both installations around.
The problem with that approach is that it breaks package managers, as the renamed files will not
be featured in the package manager's database and, consequently, will not be tracked anymore. Further, 
unless executables depending on the renamed files are modified to reflect their new path, users need
to define which executable to activate at a given time, usually through a tricky management of
symbolic links~\cite{RHEL?}.

Besides the historical lack of support for seamlessly managing the installation and runtime
execution of programs that depend on conflicting packages, operating systems face another problem:
handling dependencies distributed by third-party software. A typical case is that of programming
language modules: each programming language community packs and hosts modules according to their own
format and enables their discovery and installation through special programs that many times are not
properly integrated with the regular package manager. Examples include PIP, which manages Python
packages, LuaRocks, which controls the installation and removal of Lua modules, CPAN, to control the
installation of Perl packages, and RubyGems, which does the same for Ruby packages. Orchestrating
dependencies managed by such a variety of programs can be quite complex.

Many approaches have emerged through the years to work around the problem. In common, they propose
encapsulation techniques that let software vendors ship their programs along with their dependencies
in a single file and that let users execute said software effortlessly. Virtual machine images
became a very popular distribution format in the early 2000's~\cite{Who?}, followed more recently by
container images~\cite{Docker} that became possible thanks to advancements in microprocessor
architectures~\cite{Intel} and in operating systems support for virtualization~\cite{KVM}. In spite
of their success, an ever growing number of mundane programs gained their containerized version,
even when those have no dependencies other than basic system libraries. As a result, users end up
with several full-blown copies of operating system images that need to be managed in addition to
the regular package management for fundamental operating system services.

This paper presents a novel approach to the management and execution of programs that depends
solely on changes to the layout of the filesystem hierarchy and on private process namespaces.
Our lightweight proposal to filesystem virtualization has been fully integrated into a Linux-based
distribution and includes support for both basic programs (such as those typically managed by RPM
and DEB) and those managed by third-party software such as PIP and LuaRocks.

The remaining of the paper is organized as follows. Section~\sec{foo} presents \ldots

\section{Prior art}
Explain how Docker~\cite{Docker} works.

Snap and Flatpak.

Explain how GNU Guix~\cite{Guix} works. There is a whole distribution (GuixSD) built around it.
Guix provides support for the installation of multiple programs by installing them on unique
prefixes that include the program name and a hash string that is calculated over the program's
build inputs. For instance, two distinct versions of the Python interpreter could be installed
under /gnu/store/[sha256-hash]-python-2.7.12 and /gnu/store/[sha256-hash]-python-3.6.0, each
of which having their own bin, include, lib, and share subdirectories. \ldots
Explain that there is no support for virtualization of programming language modules. More details
here: https://github.com/pjotrp/guix-notes/blob/master/PYTHON.md
https://www.gnu.org/software/guix/blog/2018/tarballs-the-ultimate-container-image-format/


\section{Container-free filesystem virtualization}

\subsection{The layout of the filesystem tree}
Our approach to virtualization came up as a natural evolution of the filesystem tree proposed
by GoboLinux~\cite{GoboLinux}. \ldots

Talk about Resources/Dependencies
Talk about Resources/Environment
Talk about Resources/Architecture

\subsection{Augmented filesystem tree with AlienVFS}
AlienVFS is a component of GoboLinux that exposes packages installed by third-party software
(i.e., PIP, CPAN, LuaRocks, and others) as virtual program entries under /Programs. \ldots

\subsection{Filesystem virtualization with Runner}
We scan the dependencies of a given program, stated by that program's Resources/Dependencies
file, and then construct an overlay on top of /System/Index that is only visible to the process
that executes that program. \ldots


\section{Thoughts and comparison with the prior art}
Things to note: containers can limit access to underlying hardware infrastructure, segment the
network, and so on. They provide a complex infrastructure that lets system administrators to
confine the execution of programs in a way that eases the control of available resources.

We have developed and deployed our system on a Linux-based distribution.

\section{Conclusion}
This paper presented a container-free infrastructure to manage and run programs that have
conflicting dependencies. 
